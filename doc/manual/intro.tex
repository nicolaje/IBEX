\chapter{Introduction}


\ibex is a C++ constraint programming library over the reals for non-linear system solving and global optimization.

It is an academic open-source project, distributed under the LGPL license.

\ibex can be viewed as a multi-layer software, each layer providing a simple and clean interface, corresponding to a certain usage.
You can perform basic interval operations at the lowest layer or design your own branch \& bound strategy at the highest layer.

%This documentation is broken in two parts. 

%The first part is dedicated to end users.

The first chapter deals with the basic interval arithmetic operations.
The second chapter shows how to model your problem using \ibex.
In particular, you will learn how to create a system of equations or an optimization problem. Modeling basically amounts to
define a (vector-valued) function and the syntax here shares some similarities with Matlab.

In the third chapter, you will learn how to use \ibex for automatically calculating domains (under the form of boxes) where some 
properties with respect to your model are guaranteed to be satisfied. 
Typically, you can get a set of small boxes encompassing all the solutions of a system of equations.

%The second part, designed for advanced users,  discloses more details on the contractor layer.
