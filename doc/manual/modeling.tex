\chapter{Modeling your Problem}\label{chap:mod}

%In this chapter, you will learn how to create a problem

\section{Introduction}

The purpose of this chapter is to show how to create and manipulate
objects corresponding to the mathematical concepts of
{\it variable}, {\it function}, {\it constraint} and {\it system}.

We talk about {\it modeling} as these objects mathematically model the concrete problem you are faced to.

\subsection{C++ versus \quimper}

There are two possible alternatives for modeling.
You can either:
\begin{enumerate}
\item write C++ code. Variables, functions, constraints
and systems are C++ object that you declare yourself
and build by calling the constructors of the corresponding classes
\item write all these basic mathematic data in a text file, following
the (very intuitive) \quimper syntax. All these data are loaded
simultaneously and stored in a single \hcf{System} object.
\end{enumerate}

In both cases, you will access and use the data in the same way.
For instance, you will calculate the interval derivative of a function
by the same code, would it be created in your C++ program or loaded
from a \quimper file.

The chapter is organized as follows: we present
each concept (variable, function, etc.) in turn and
each time explain how objects are created in C++.

All the \quimper syntax is given afterwards, in a separate section.
 
\subsection{What we mean by ``variable'' and ``function''}

Let us rule out a potential ambiguity.

Since we are in the C++ programming language, the term {\it variable} and
{\it function} already refers to something precise. For instance, the following
piece of code introduces a {\it function} \cf{sum} and a {\it variable} \cf{x}:
\begin{lstlisting}
int sum(int x, int y) { 
  return x+y;
}

int x=2;
\end{lstlisting}
The variable $x$ may represent, say, the balance of a bank account.
The account number is what we call the {\it semantic} of $x$, that is, what $x$ is supposed
to represent in the user's mind. So, on one side,
we have {\it what we write}, that is, a program with variables and functions,
 and on the other side, {\it what we represent}, that is, concepts 
like a bank account.

With \ibex, we write programs to represent mathematical concepts
that are also called {\it variables} and {\it functions}.
The mapping $(x,y)\mapsto \sin(x+y)$ is an example of function that
we want to represent. It shall not be confused with the function \cf{sum}
above.

To avoid ambiguity, we shall talk about {\it mathematical}
variables (resp. functions) versus {\it program} variables (resp. functions).
We will also use italic symbol like $x$ to denote a mathematical variable
and postscript symbols like \cf{x} for program variables.
In most of our discussions, variables and functions will refer
to the mathematical objects so that the mathematical meaning will be the implicit one. 

\section{Variables}

Before telling you which class represents mathematical variables, let us say first that
this class does not play a big role.
%The reason is that variables are only used to build functions.
Indeed, the only purpose of declaring a variable $x$ in \ibex is
for building a function right after, like $f:x\mapsto x+1$.
Functions play, in contrast, a big role.

In other words, $x$ is nothing but a syntaxic leaf in the expression 
of the function. In particular, a variable is not a slot for
representing domain. % (as it could be the case). %This is an important difference with other 
E.g, if you want to calculate the range of $f$ for $x\in [0,1]$,
you just call a (program) function \cf{eval} with a plain box in argument.
It's just as if $f$ was the function that takes one argument and
increment it, whatever the name of this argument is.

Once $f$ has been built, we can almost say that $x$ is no longer useful.
Variables must be seen only as temporary objects, in the process of function construction.

Before going on, let us slightly moderate this point.
We have assumed here that, as a user of \ibex the operations you are interested in are: {\it evaluate $f$ on a box, 
calculate $f'$ on a box, solve $f(x)=0$} and so on. All these operations can be qualified as numerical: they take
intervals and return intervals. You don't need to deal again with the expression of the function, once built.
But if you need to handle, for any reason, the symbolic form of the function then you have to inspect the syntax
and variables appear again.

\subsection{Dimensions and ordering}

A variable does not necessarily represent a single real value.
It can also be a vector, a matrix or an array-of-matrices. So each variable has some associated information about
its dimension(s). 

One can, e.g., build the following function 
$$\begin{array}{cccc}
f: & \Rset^2\times\Rset^3 & \to & \Rset\\
   &  (x,y) & \mapsto & x_1\times y_1+x_2\times y_2 - x_3
\end{array}.$$

In this case, $x$ and $y$ are vector variables with 2 and 3 components respectively.

From the user standpoint, the function $f$ (once built) is ``flattened'' to a mapping from $\Rset^5$ to $\Rset$.
Each numerical function (eval, etc.) expects a 5-dimensional box as argument.

The way intervals are mapped to the variables components follows a straightforward ordering:
everytime we call a (program) function of $f$ with the box $[b]=([b]_1,\ldots,[b]_5)$ in argument, we simply enforce
$$x\in[b]_1\times[b]_2 \quad \mbox{and} \quad y\in[b]_3\times[b]_4\times[b]_5.$$

If you don't want to create functions in C++, you can skip now to Section \ref{sec:mod-func}.

\subsection{Class name and fields}

As we have just said, variables are just symbols in expression. For this reason,
they are represented by a classe named \hcf{ExprSymbol}.
In fact, there is also another class we introduced for convenience, called \hcf{Variable}.
Without going further into details, the latter must be seen as a kind of ``macro'': a \hcf{Variable} object
generates \hcf{ExprSymbol} objects. This macro is only useful if you build variables in C++ (see \S\ref{sec:mod-var-cpp}).

Once built, a variable is always typed \hcf{ExprSymbol}.

If \cf{x} is an \hcf{ExprSymbol} object, you can obtain the information about its dimensions via \cf{x.dim}.
The \hcf{dim} field is of type \hcf{Dim}, a class that simply contains 3 integers (one for each dimension, see
the API for further details).

Finally, a variable also have a name, that is only useful for displaying. It is a 
regular C string (\cf{char*}) stored in the field \hcf{name}.


%\subsection{Vector and matrix variables}



\subsection{Creating variables (in C++)}\label{sec:mod-var-cpp}

The following piece of code creates a variable \cf{x} and prints it.

\begin{lstlisting}
  Variable x;
  cout << x << endl;
\end{lstlisting}

The first instruction creates a (program) variable \cf{x}. It is initialized by default, since
no argument are given here to the constructor.
By default, the variable is real (or {\it scalar}), meaning it is not a vector nor a matrix. 
Furthermore, the (mathematical) variable has a name that is automatically
generated. Of course, the name of the mathematical variable does not necessarily correspond to the name of the 
program variable.
For instance, \cf{x} is the name of a C++ variable but the corresponding 
mathematical variable is named {\it \_x\_0}.
The second instruction prints the name of the mathematical variable on the standard output:

\begin{lstlisting}
_x_0
\end{lstlisting}

It is possible to rename variables, see below. %\S\ref{sec:mod-var-name}.

\subsubsection{Vector and matrix variables}\label{sec:mod-var-vec}
To create a $n$-dimensional vector variable, just
give the number $n$ as an arguement to the constructor:

\begin{lstlisting}
  Variable y(3);   // creates a 3-dimensional vector
\end{lstlisting}

To create a $m\times n$ matrix, give $m$ (number of rows) and $n$ (number of columns) as arguments:

\begin{lstlisting}
  Variable z(2,3);   // creates a 2*3-dimensional matrix
\end{lstlisting}

We can go like this up to 3 dimensional arrays:

\begin{lstlisting}
  Variable t(2,3,4);   // creates a 2*3*4-dimensional array
\end{lstlisting}


\subsubsection{Renaming variables}\label{sec:mod-var-name}
Usually, you don't really care about the names of mathematical variables since you handle
program variables in your code.
However, if you want a more user-friendly display, you can specify
the name of the variable as a last argument to the constructor.

In the following example, we create a scalar, a vector and a matrix variable each
time with a chosen name.

\begin{lstlisting}
  Variable x("x");   // creates a real variable named "x"
  Variable y(3,"y");   // creates a vector variable named "y"
  Variable z(2,3,"z");   // creates a matrix variable named "z"
  cout << x << " " << y << " " << z << endl;
\end{lstlisting}

Now, the display is:
\begin{lstlisting}
x y z
\end{lstlisting}


\section{Functions}\label{sec:mod-func}

Mathematical functions are represented by objects of the class \hcf{Function}.

These objects are very easy to build. Either refer to \S\ref{sec:mod-func-cpp}
or \S\ref{sec:mod-func-quimper}.

\subsection{Evaluation (forward computation)}

We show now how to calculate the image of a box by a function $f$ or, in short, how
to perform an {\it interval evaluation}.

We assume that the function \cf{f} has been created to represent $(x,y)\mapsto \sin(x+y)$ (see the previous paragraph).

We start by building a box (as explained in Chapter \ref{chap:arith}).
The box must have as many components as the function has arguments, here, 2.

Then we simply call \hcf{f.eval(...)} to get the image of the box by \cf{f}:

\begin{lstlisting}
  double _box[][2]= {{1,2},{3,4}};
  IntervalVector box(2,_box);         // create the box ([1,2];[3,4])

  cout << "initial box=" << box << endl;
  cout << "f(box)=" << f.eval(box) << endl; 
\end{lstlisting}

\subsection{Projection (backward computation)}
	
\subsection{Gradient}

\subsection{Jacobian and Hansen's matrix}

\subsection{Creating functions (in C++)}\label{sec:mod-func-cpp}

The following piece of code creates the function
$(x,y)\mapsto \sin(x+y)$:

\begin{lstlisting}	
  Variable x("x");
  Variable y("y");
  Function f(x,y,sin(x+y));
  cout << f << endl;
\end{lstlisting}

The display is:
\begin{lstlisting}
_f_0:(x,y)->sin((x+y))
\end{lstlisting}
%_f_0:(_x_0,_x_1)->sin((_x_0+_x_1))

\subsubsection{Renaming functions}

By default, function names are also generated. But you can also set your own function name, as the last argument of the constructor:
\begin{lstlisting}
Function f(x,y,sin(x+y),"f");
\end{lstlisting}

\subsubsection{Allowed symbols}

The following symbols are allowed in expressions:
\begin{verbatim}
sign, min, max,
sqr, sqrt, exp, log, pow, 
cos, sin, tan, acos, asin, atan,
cosh, sinh, tanh, acosh, asinh, atanh
atan2
\end{verbatim}

Power symbols \cf{^} are not allowed. You must
either use \cf{pow(x,y)}, or simply \cf{sqr(x)} for the square function.

Here is an example of the distance function between (\cf{xa},\cf{ya}) and
(\cf{xb},\cf{yb}):

\begin{lstlisting}
  Variable xa,xb,ya,yb;
  Function dist(xa,xb,ya,yb, sqrt(sqr(xa-xb)+sqr(ya-yb)));
\end{lstlisting}

\subsubsection{Functions with vector variables}

If variables are vectors, you can refer to the component
of a variable using square brackets. Indices start by 0,
following the convention of the C language.

We rewrite here the previous distance function using 2-dimensional
variables \cf{a} and \cf{b} instead:
\begin{lstlisting}
  Variable a(2);
  Variable b(2);
  Function dist(a,b,sqrt(sqr(a[0]-b[0])+sqr(a[1]-b[1])),"dist");
\end{lstlisting}

\subsubsection{Composition}

You can compose functions. Each argument of the called function can be substitued
by a variable, a subexpression or a constant value.

For instance, we can define a function "foo" that associates to
a point \cf{x} the distance between \cf{x} and a fixed point $(1,2)$,
using the generic distance function defined in the previous paragraph:

\begin{lstlisting}
  Vector pt(2);
  pt[0]=1;
  pt[1]=2;

  Variable x(2);
  Function f(x,dist(x,pt),"foo");
\end{lstlisting}

\subsubsection{Vector-valued functions}

To define a vector-valued function, the \hcf{Return} keword allows
you to list the function's components.

For instance, we can define the function that associates to $x$ the 
respective distances between two fixed points \cf{pt1} and \cf{pt2}:

\begin{lstlisting}	
	Variable x(2,"x");
	Variable pt(2,"p");
	Function dist(x,pt,sqrt(sqr(x[0]-pt[0])+sqr(x[1]-pt[1])),"dist");

	Vector pt1=Vector::zeros(2);
	Vector pt2=Vector::ones(2);

	Function f(x,Return(dist(x,pt1),dist(x,pt2)),"f");

	cout << f << endl;
\end{lstlisting}

The display is as folllows. Note that constant values like 0 are automatically replaced
by degenerated intervals (like [0,0]):
\begin{verbatim}
f:(x)->(dist(x,([0,0] ; [0,0])),dist(x,([1,1] ; [1,1])))
\end{verbatim}

\section{Constraints}

\subsection{Creating constraints (in C++)}

\section{Systems}

A {\it system} in \ibex is a set of constraints with, optionnaly, a goal function to minimize.
One is usually interested in solving the system while minimizing the criterion, if any.

For this reason, a system is not as simple as a collection of {\it any} constraints:
each constraint must exactly relates the same set of variables. And this set must
also coincide with that of the goal function.
Many algorithms of \ibex are based on this assumption.
This is why they requires a system as argument (and not just an array of constraints).
This makes systems a central concept in \ibex.

A system is an object of the \hcf{System} class. This object is made of several fields
that are detailed below. %We take as example the following system:

\begin{itemize}
\item \cf{const int} \hcf{nb_var}: the total number of variables or, in other words, the
{\it size} of the problem. This number is basically the sum of all variables components. For instance,
if one declares a variable $x$ with 10 components and a variable $y$ with 5, the value of this field
will be $15$.
\item \cf{const int} \hcf{nb_ctr}: the number of constraints
\item \cf{Function*} \hcf{goal}: a pointer to the goal function. If there is no goal function, this
pointer is \cf{NULL}.
\item \cf{Function} \hcf{f}: the (usually vector-valued) function representing the constraints. 
For instance, if one defines three constraints: $x+y\leq0$ and $x-y=1$ and $x-y\geq0$, the function f will be 
$(x,y)\mapsto (x+y,x-y-1,x-y)$. Note that the constraints are automatically transformed so that the right side 
is 0 but, however, without changing the comparison sign. It is however possible to {\it normalize} a system so that
all inequalities are defined with the $\le$ sign (see \S\ref{sec:mod-sys-transfo}).
\item \cf{IntervalVector} \hcf{box}: when a system is loaded from a file (see \S\ref{sec:mod-sys-load}),
a initial box can be specified. It is contained in this field.
\item \cf{Array<NumConstraint>} \hcf{ctrs}: the array of constraints. The \cf{Array} class of \ibex can
be used as a regular C array.
\end{itemize}

\subsection{Copy and transformation}\label{sec:mod-sys-transfo}

\subsection{Auxiliary functions}\label{sec:mod-sys-auxfunc}

\subsection{Creating systems (in C++)}

The first alternative for creating a system is to do it programmatically, that is, directly in your C++ program.
Creating a system in C++ resorts to a temporary object called a {\it system factory}. The task is done in a few simple steps:
\begin{enumerate}
\item declare a new system factory (an object of \hcf{SystemFactory})
\item add variables in the factory using \hcf{add_var}.
\item (optional) add the expression of the goal function using \hcf{add_goal}
\item add the constraints using \hcf{add_ctr}
\item create the system simply by passing the factory to the constructor of \cf{System}
\end{enumerate}

Here is an example:

\begin{lstlisting}
  Variable x,y;

  SystemFactory fac;
  fac.add_var(x);
  fac.add_var(y);
  fac.add_goal(x+y);
  fac.add_ctr(sqr(x)+sqr(y)<=1);

  System sys(fac);
\end{lstlisting}

\section{The \quimper syntax}\label{sec:mod-sys-load}

Entering a system programmatically is usually not very convenient.
You may prefer to separate the model of the problem from the algorithms
you use to solve it. In this way, you can run the same program with different
variants of your model without recompiling it each time.

\ibex provides such possibility. You can directly load a system from a (plain text) input file.

Here is a simple example. Copy-paste the text above in a file named, say, {\tt problem.txt}. 
The syntax talks for itself:

\begin{verbatim}
Variables
  x,y;

Minimize
  x+y;

Constraints
  x^2+y^2<=1;
end
\end{verbatim}

Then, in your C++ program, just write:

\begin{lstlisting}
System sys("problem.txt");
\end{lstlisting}

and the system you get is exactly the same as in the previous example.

Next sections details the mini-language of these input files. 

\subsection{Overall structure}
First of all, the input file is a sequence of declaration blocks that must respect the following order:
\begin{enumerate}
\item (optional) constants
\item variables
\item (optional) auxiliary functions
\item (optional) goal function
\item constraints
\end{enumerate}

Next paragraph gives the basic format of number and intervals.
The subsequent paragraphs detail each declaration blocks.

\subsection{Real and Intervals}
A real is represented with the usual English format, that is
with a dot separating the integral from the decimal part,
and, possibly, using scientific notation.

Here are some valid examples of reals in the syntax:
\begin{verbatim}
0
3.14159
-0.0001
1.001e-10
+70.0000
\end{verbatim}
 
An interval are two reals separated by a comma
and surrounded by square brackets. The special symbol
{\tt oo} represents the infinity $\infty$.
Note that, even with infinity bounds, the brackets
must be squared (and not parenthesis as it should be since the
bound is open). Here a some examples:

\begin{verbatim}
[0,1]
[0,+oo]
[-oo,oo]
[1.01e-02,1.02e-02]
\end{verbatim}

\subsection{Constants}
Constants are all defined in the same declaration block, 
started with the \cf{Constants} keyword.
A constant value can depends on other (previously defined) constants value. Example:

\begin{verbatim}
Constants
  pi=3.14159;
  y=-1.0;
  z=sin(pi*y);
\end{verbatim} 

You can give a constant an interval enclosure rather than a single fixed value.
This interval will be embedded in all subsequent computations.
In the previous example, we can give \cf{pi} a valid enclosure as below:

\begin{verbatim}
Constants
  pi in [3.14159,3.14160];
  y=-1.0;
  z=sin(pi*y);
\end{verbatim}

Constants can also be vector, matrices or array of matrices.
In this case, the dimension is specified in brackets
right after the name. For instance \cf{x[2]} means that 
$x$ is a vector with 2 components. If the first component is equal
to $0$ and the second to $1$, we write: 
\begin{verbatim}
Constants
x[2] = [0; 1]
\end{verbatim}

{\bf important remark}. Do note confuse
\cf{[0;1]} with \cf{[0,1]}
\begin{itemize}
\item \cf{[0;1]} is a 2-dimensional vector of two reals, namely 0 and 1
\item \cf{[0,1]} is the 1-dimensional interval $[0,1]$
\end{itemize}
The difference is ``\cf{;}'' versus ``\cf{,}''.

Of course, you mix vector with intervals. For instance:
{\tt [[-oo,0];[0,+oo]]} is a vector of $2$ intervals, $(-\infty,0)$ and $(0,+\infty)$.

To declare matrix constant, add an extra dimension in brackets:
\begin{verbatim}
Constants
M[3][2] = [[0; 0] ; [0; 1]; [1; 0]]
\end{verbatim}
This will create the constant matrix \cf{M} with 3 rows and 2 columns equal to
$$\begin{pmatrix}
0 & 0 \\ 0 & 1 \\ 1 & 0
\end{pmatrix}$$


The following table summarizes the possibility for declaring constants
through different examples.

\begin{tabular}{l|l}
\hline
{\tt x in [-oo,0]} & declares a constant $x$ with domain $(-\infty,0]$ \\
{\tt x in [0,1]} & declares an constant $x$ with domain $[0,1]$ \\
{\tt x in 0} & declares a (interval) constant $x$ with domain $[0,0]$ \\
{\tt x = 0} & declares a (real) constant $x$ equal to $0$ \\
{\tt x = 100*sin(0.1)} & declares a constant $x$ equal to $100\sin(0.1)$ \\
{\tt x[10] in [-oo,0]} & declares $10$ constants $x[1],\ldots,x[10]$, each with domain $(-\infty,0]$ \\
{\tt x[2] in [[-oo,0];[0,+oo]]} & declares $2$ constants $x[1]\in(-\infty,0]$ and $x[2]\in[0,+\infty)$ \\
{\tt x[10] in [0,1]} & declares $10$ constants $x[1],\ldots,x[10]$, each with domain $[0,1]$ \\
{\tt x[3][3] in} &  \multirow{4}{*}{declares a constrant matrix 
$x=\begin{pmatrix}
\!\,[0,1] & 0 & 0 \\
0 & [0,1]& 0 \\
0 & 0 & [0,1] 
\end{pmatrix}$.} \\
\!\,{\tt [[[0,1];0;0];} & \\
\!\,{\tt [0;[0,1];0];} & \\
\!\,{\tt [0;0;[0,1]]]} &\\
\multirow{2}{*}{\tt x[10][5] in [0,1]} & declares $50$ constants $x[1][1],\ldots,x[1][5],\ldots,x[10][1],\ldots,x[10][5]$, \\
& each with domain $[0,1]$.\\
\end{tabular}

It is possible to define up to three dimensional vectors. 

\subsection{Functions}\label{sec:mod-func-quimper}

When the constraints involve the same expression repeatidly, it may be
convenient for you to put this expression once for all in a separate auxiliary
function and to call this function.

Assume for instance that your constraints intensively use the following expression
$$\sqrt{(x_a-x_b)^2+(y_a-y_b)^2)}$$
where $x_a,\ldots y_b$ are various sub-expressions, like:
\begin{verbatim}
sqrt((xA-1.0)^2+(yA-1.0)^2<=0;
sqrt((xA-(xB+xC))^2+(yA-(yB+yC))^2=0;
...
\end{verbatim}

You can declare the distance function as follows.
\begin{verbatim}
function d=distance(xa,ya,xb,yb)
 d=sqrt((xa-xb)^2+(ya-yb)^2
end
\end{verbatim}
Note that \cf{d} corresponds to the returned value of the function,
as in the Matlab syntax. We will call it the {\it return variable}.
You will then be able to simplify the writing of constraints:
\begin{verbatim}
distance(xA,1.0,yA,1.0)<=0;
distance(xA,xB+xC,yA,yB+yC)=0;
...
\end{verbatim}

As you may expect, this will result in the creation of
a \cf{Function} object (see \S\ref{sec:mod-func}) that
you can access from your C++ program. This will be explained
in \ref{sec:mod-sys-auxfunc}.

A function can return a single value, a vector
or a matrix. Similarly, it can take real, vectors or matrix arguments.
You can also write some minimal ``code'' inside the function before
assigning the final expression to the return variable. This
code is however limited to be a sequence of assignments.

Let us now illustrate all this with a more sophisticated example.
We write below the function that calculates the rotation matrix
from the three Euler angles, $\phi$, $\theta$ and $\psi$ :

$R : (\phi,\psi,\theta) \mapsto$
{\scriptsize
$$\begin{pmatrix}
\cos(\theta)\cos(\psi) & -\cos(\phi)\sin(\psi)+\sin(\theta)\cos(\psi)\sin(\phi) & \sin(\psi)\sin(\phi)+\sin(\theta)\cos(\psi)\cos(\phi)\\
\cos(\theta)\sin(\psi) & \cos(\psi)\cos(\phi)+\sin(\theta)\sin(\psi)\sin(\phi) & -\cos(\psi)\sin(\phi)+\sin(\theta)\cos(\phi)\sin(\psi)\\
-\sin(\theta) & \cos(\theta)\sin(\phi) & \cos(\theta)\cos(\phi);
\end{pmatrix}
$$}

As you can see, there are many occurrences of the same subexpression
like $\cos(\theta)$ so a good idea for both readibility and (actually) efficiency
is to precalculate such pattern and put the result into an intermediate variable.

Here is the way we propose to define this function:

\begin{verbatim}
/* Computes the rotation matrix from the Euler angles: 
   roll(phi), the pitch (theta) and the yaw (psi)  */
function R[3][3]=euler(phi,theta,psi)
  cphi   = cos(phi);
  sphi   = sin(phi);
  ctheta = cos(theta);
  stheta = sin(theta);
  cpsi   = cos(psi);
  spsi   = sin(psi);
  
  R[1][1]=ctheta*cpsi;
  R[1][2]=-cphi*spsi+stheta*cpsi*sphi;  
  R[1][3]=spsi*sphi+stheta*cpsi*cphi;
  R[2][1]=ctheta*spsi;    
  R[2][2]=cpsi*cphi+stheta*spsi*sphi;   
  R[2][3]=-cpsi*sphi+stheta*cphi*spsi;
  R[3][1]=-stheta;        
  R[3][2]=ctheta*sphi;                  
  R[3][3]=ctheta*cphi;
end
\end{verbatim}

{\bf Remark}. Introducing temporary variables like \cf{cphi} amouts to build a DAG instead of
a tree for the function expression. It is also possible (and easy) to build a DAG when you directly create
a \cf{Function} object in C++ (to be documented).

\subsection{Variables}

\subsection{Constraints}
\subsubsection{Loops}

You can resort to loops in a Matlab-like syntax to define constraints. Example:

\begin{verbatim}
Variables
  x[10];

Constraints
  for i=1:10;
    x[i] <= i;
  end
end
\end{verbatim}

\subsection{Difference with C++}

\begin{itemize}
\item Indices start by $1$ instead of $0$
\item You can use the \^{} symbol
\end{itemize}

